%% LyX 2.2.3 created this file.  For more info, see http://www.lyx.org/.
%% Do not edit unless you really know what you are doing.
\documentclass[english]{article}
\usepackage[T1]{fontenc}
\usepackage[latin9]{inputenc}
\usepackage{babel}
\begin{document}

\title{EECS/CSE 31L Midterm Project}

\author{\emph{32-Bits Enhanced ALU Design }}

\author{Leo Leal, Eliseo Nunez, Donald Isbell, Manpreet Singh }

\date{10/30/2017}
\maketitle

\part*{Introduction}

\paragraph*{The vision of this project was to complete the design of an 32-bits
enhanced ALU (Arithmetic Logic Unit) to perform a variety of functions
beyond that of standard addition or subtraction. Upon undertaking
this task we took upon ourselves to set\textmd{ }goals for ourselves
in order to create the most efficient and relevant ALU as possible.
The first of our goals was modualbility; a focus on this idea allowed
for our code to be able to be expanded and contracted as needed, as
well being able to divide the ALU into smaller modules allows for
us to be able to expedite debugging and implement non-problematic
optimization. Our second goal was efficiency; although not particularly
problematic in this specific project, we strove to use a minimal amount
of variables and through the use of deductive logic, minimize the
amount of gates required thus decreasing the amount of time for the
circuit to complete. Our last goal for the project was readability;
which is slightly tied into our previous goals because at the end
of the day in order for our code to be reusable it needed to be clear
and concise so that if anyone picked up the code, even with a lack
of comments they would be able to understand what would be happening
in the circuit. Through our focus on these three goals we would be
able to complete this project and design an enhanced ALU that would
surpass expectations.}

\part*{Design }

\subsection*{Initialization}

\paragraph*{The overall design over our code includes 3 inputs and 5 outputs.
The inputs include the two 32 bit numbers and the operation code which
determines what answer will be provided in the output of the circuit.
As well, we have the 4 flags as outputs with the carryout , overflow,
zero, and sign flags. There are twelve functions that the circuit
performs whenever numbers are present in the input. There is only
and output however when there is the presence of an the operation
code that indicates which function result is to be provided. }

\subsection*{Function/Processing}

\paragraph*{After initialization, all twelve functions begin to process the inputs.
All the functions work in their standard form, receiving the two numbered
inputs as well as the indexed flag outputs. All of these modules are
separated into their own source files and are called into the main
ALU. All of the functions store their answers into an output that
leads to our selection algorithm. Our selection algorithm goes through
the operation code inputs from least significant bit to most significant
combing through all of the 12 possible outcomes in 12 steps by using
the assignment operators. }

\subsection*{Output}

\paragraph*{There are a total of four Multiplexers that we use in this code,
one for each of the flags. The multiplexer is effective for determining
quickly which one of the indexed values we need for the flags. The
overall output is determined through those 12 lines of code mentioned
in the previous statement that saved our circuit from having to have
32 twelve to one multiplexers which would be an extreme amount of
gates. Instead we were able to run all the values through strategically
placed gates.}

\part*{Testing}

\paragraph*{When it came to testing our code we ran it through many cases, focusing
especially on extraneous cases that could possibly cause the most
errors especially when it came to our increment in decrements circuits.
We experimented to make sure nothing would happen if simply one number
were forced, or if there would be any issues with extremely large
or small numbers that would be represented in our machine. Overall
very few bugs were ever found, and of those that were, we discovered
that they had old code that had not been updated from the server yet
which thus left to faulty testing. We threw out the old info and once
the issue was corrected we began the process again. By the end of
our efforts we were unable to determine any bugs in the code as far
as we could see. It was efficient and accurate just as we had intended
it to be. By the end of our investigation for issues over 50 tests
had been performed on the code (some of which you can view in the
folders provided), all said and done we had not yet found any issues
with the code.}

\section*{Conclusion}

\paragraph*{At the beginning of this project we had our problem before us, developing
an enhanced ALU capable of performing twelve of these specific functions
on 32 bit numbers. In wake of this issue we decided to set for ourselves
three goals: Moduability, Efficiency and Readability. From our tremendous
amount of source files of which are all separated into modules of
which could be used in other projects. We developed a code that would
be able to grow and change into whatever we may need it to be in the
future. With our efficient logic in determining a way in which we
were able to cycle through much more information with less gates thus
creating a machine that runs cooler and faster. Through viewing all
of our .sv files it you can clearly see the path we took with our
project and can understand all that is going on in this ALU even if
you don't have an extensive knowledge of systemverilog. Lastly through
developing a code that had no issues in our debugging phases we can
firmly say without a doubt that this project has been a success. We
planned what we wanted to put in and get out of the project, constructed
a way to complete our goal and achieve our success.}
\end{document}
